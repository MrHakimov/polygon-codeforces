\begin{problem}{Valeriy and Deque}{standard input}{standard output}{6 seconds}{256 megabytes}

Recently, on the course of algorithms and data structures, Valeriy learned how to use a deque. He built a deque filled with $n$ elements. The $i$-th element is $a_i$ ($i$ = $1, 2, \ldots, n$). He gradually takes the first two leftmost elements from the deque (let's call them $A$ and $B$, respectively), and then does the following: if $A > B$, he writes $A$ to the beginning and writes $B$ to the end of the deque, otherwise, he writes to the beginning $B$, and $A$ writes to the end of the deque. We call this sequence of actions an operation.

For example, if deque was $[2, 3, 4, 5, 1]$, on the operation he will write $B=3$ to the beginning and $A=2$ to the end, so he will get $[3, 4, 5, 1, 2]$.

The teacher of the course, seeing Valeriy, who was passionate about his work, approached him and gave him $q$ queries. Each query consists of the singular number $m_j$ $(j = 1, 2, \ldots, q)$. It is required for each query to answer which two elements he will pull out on the $m_j$-th operation.

Note that \textbf{the queries are independent} and for each query the numbers $A$ and $B$ should be \textbf{printed in the order in which they will be pulled out of the deque}.

\textit{Deque} is a data structure representing a list of elements where insertion of new elements or deletion of existing elements can be made from both sides.


\InputFile
The first line contains two integers $n$ and $q$ ($2 \leq n \leq 10^5$, $0 \leq q \leq 3 \cdot 10^5$) "--- the number of elements in the deque and the number of queries.
The second line contains $n$ integers $a_1$, $a_2$, ..., $a_n$, where $a_i$ $(0 \leq a_i \leq 10^9)$ "--- the deque element in $i$-th position.
The next $q$  lines contain one number each, meaning $m_j$ ($1 \leq m_j \leq 10^{18}$).

\OutputFile
For each teacher's query, output two numbers $A$ and $B$ "--- the numbers that Valeriy pulls out of the deque for the $m_j$-th operation.

\Examples

\begin{example}
\exmpfile{example.01}{example.01.a}%
\exmpfile{example.02}{example.02.a}%
\end{example}

\Note
\begin{enumerate}
Consider all 10 steps for the first test in detail:

\item $[1, 2, 3, 4, 5]$ "--- on the first operation, $A$ and $B$ are $1$ and $2$, respectively.

So, $2$ we write to the beginning of the deque, and $1$ "--- to the end.

We get the following status of the deque: $[2, 3, 4, 5, 1]$.

\item $[2, 3, 4, 5, 1] \Rightarrow A = 2, B = 3$.

\item $[3, 4, 5, 1, 2]$

\item $[4, 5, 1, 2, 3]$

\item $[5, 1, 2, 3, 4]$

\item $[5, 2, 3, 4, 1]$

\item $[5, 3, 4, 1, 2]$

\item $[5, 4, 1, 2, 3]$

\item $[5, 1, 2, 3, 4]$

\item $[5, 2, 3, 4, 1] \Rightarrow A = 5, B = 2$.
\end{enumerate}

\end{problem}
